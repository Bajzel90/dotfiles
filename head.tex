\usepackage[top=2cm, bottom=1.5cm, left=2cm, right=2cm]{geometry}

\PassOptionsToPackage{dvipsnames,svgnames,x11names}{xcolor}
\usepackage{hyperref}

% begin a new page for each section (first level header),
% we need to combine this with `-V subparagraph` when invoking pandc
% \usepackage{titlesec}
% \newcommand{\sectionbreak}{\clearpage}

% change background color for inline code in markdown files.
% The following code does not work well for long text, because the text will exceed the page boundary.
\definecolor{bgcolor}{HTML}{DADADA}
\let\oldtexttt\texttt

\renewcommand{\texttt}[1]{
  \colorbox{bgcolor}{\oldtexttt{#1}}
}


% start a new page after toc, we need to save the old command before defining
% new one to avoid recursive command calls,
% see https://tex.stackexchange.com/questions/47351/can-i-redefine-a-command-to-contain-itself
\let\oldtoc\tableofcontents
\renewcommand{\tableofcontents}{\oldtoc\newpage}


\usepackage{float}
\let\origfigure\figure
\let\endorigfigure\endfigure
\renewenvironment{figure}[1][2] {
    \expandafter\origfigure\expandafter[H]
} {
    \endorigfigure
}

\usepackage{listings}
\usepackage{xcolor}

\lstset{
    backgroundcolor=\color[RGB]{240,240,240},
    basicstyle=\footnotesize\ttfamily\linespread{1.1},  % the size of the fonts that are used for the code; posssible values are (\ttfamily, \footnotesize, etc.)
    breaklines=true,                % sets automatic line breaking
    breakatwhitespace=true,         % sets if automatic breaks should only happen at whitespace
    breakautoindent=true,
    breaklines=true,
    captionpos=b,
    commentstyle=\color[rgb]{0.56,0.35,0.01}\itshape,
    escapeinside={\%*}{*)},
    frame=single,	                 % adds a frame around the code
    framesep=8pt,
    keywordstyle=\color[rgb]{0.13,0.29,0.53}\bfseries,
    linewidth=\textwidth,
    numbers=none,                    % where to put the line-numbers; possible values are (none, left, right)
    numbersep=5pt,                   % how far the line-numbers are from the code
    numberstyle=\tiny\color{gray},   % the style that is used for the line-numbers
    rulecolor=\color[RGB]{220,220,220}, % the frame color; we prefer slightly-darker than the background color
    showspaces=false,                % show spaces everywhere adding particular underscores; it overrides 'showstringspaces'
    showstringspaces=false,          % underline spaces within strings only
    showtabs=false,                  % show tabs within strings adding particular underscores
    stepnumber=2,                    % the step between two line-numbers. If it's 1, each line will be numbered
    stringstyle=\color[rgb]{0.31,0.60,0.02},
    tabsize=4,                       % sets default tabsize to 2 spaces
    xleftmargin=8pt,                 % use the same value as framesep
    xrightmargin=8pt,                % use the same value as framesep
}

% block quote

\usepackage{tcolorbox}
\newtcolorbox{myquote}{colback=red!5!white, colframe=red!75!black}
\renewenvironment{quote}{\begin{myquote}}{\end{myquote}}


% bullets

\usepackage{enumitem}
\usepackage{amsfonts}

% level one
\setlist[itemize,1]{label=$\bullet$}
% level two
\setlist[itemize,2]{label=$\circ$}
% level three
\setlist[itemize,3]{label=$\circ$}

% hypen

\exhyphenpenalty=10000
\hyphenpenalty=10000

\raggedright



% Section style

\usepackage{silence}
\WarningFilter[temp]{latex}{Command \underbar  has changed.}
\WarningFilter[temp]{latex}{Command \underline  has changed.} 

\usepackage{sectsty}
\DeactivateWarningFilters[temp]

\sectionfont{\clearpage}
\sectionfont{\LARGE\clearpage}
\subsectionfont{\clearpage}
\subsectionfont{\LARGE\clearpage}

% Set space

\usepackage{setspace}
\setstretch{1.2}
